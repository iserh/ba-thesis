\begin{abstract}
  In dieser Arbeit werden verschiedene Data Augmentation Ansätze basierend auf Variational Autoencodern untersucht. Im Fokus stehen die Analyse des erzeugten Latent-Space und die Augmentierung unterschiedlicher Datensätze, wie MNIST, CelebA und PROBEN1. Über reduzierte Varianten dieser Datensätze werden die vorgestellten Methoden auch in Few-Shot Learning Szenarios evaluiert. Die Ergebnisse offenbaren, dass Variational Autoencoder auf kleinen Datenmengen zu einer leichten Verbesserung der Performanz führen. Für größere Datensätze sind die generierten Beispiele jedoch weniger hilfreich. Hier erzielen andere Ansätze, wie Generative-Adversarial-Networks deutlich bessere Ergebnisse. Es wird sich außerdem herausstellen, dass die erzeugten Daten im Fall von Bilddaten eine starke Unschärfe aufweisen. Dennoch bietet der Variational Autoencoder viele Vorteile durch seine Latent-Space Struktur, denn dies erlaubt kontrollierbare Modifikationen der Daten.
\end{abstract}