\chapter{Fazit}
In dieser Arbeit wurden der Ansatz der Data Augmentation mit Variational Autoencodern auf unterschiedlichen Datensätzen untersucht. Die Verfahren wurden unter verschiedenen Szenarien evaluiert, welche in der Menge, Balance und Art der Daten variieren. Außerdem wurde die Repräsentation der Daten im Latent-Space durch die verschiedenen Modelle analysiert. Die wichtigsten Erkenntnisse werden im Folgenden zusammengefasst. \\

In der Evaluation konnten Schwierigkeiten des VAE Ansatzes im Umgang mit diskreten Attributen beobachtet werden. Dies machte diesen Ansatz auf numerischen Daten sehr Hyperparameter sensitiv. Außerdem wurden Probleme beim generieren neuer Daten festgestellt, da der Latent-Space eine kontinuierliche probabilistische Struktur hat. Die erzeugten Beispiele verbessern die Klassifikation daher nur geringfügig. Auf imbalancierten Datensätzen fiel die Verbesserung zusätzlich kleiner aus, da die stärker repräsentierte Klasse bessere Rekonstruktionen zur Folge hat. Im Vergleich zu den in der Arbeit von \cite{Moreno-Barea2020} genutzten generativen Methoden schneidet der VAE Ansatz schlechter ab. Im Few-Shot Szenario kann dagegen eine geringe, aber konsistente Verbesserung erreicht werden. Diese nimmt jedoch mit der Anzahl an original Beispielen ab.\\

Im Fall von Bilddaten zeigt der VAE eine , für die Modifikation von Bildern nützliche, Latent-Space Struktur. Insbesondere weiche Übergänge und organische Formen, wie Gesichtszüge, werden gut repräsentiert und erlauben Augmentierungen. Ein höher gewichteter KL-Term führte außerdem zu einer besseren Dekorrelation unterschiedlicher Merkmale in den Dimensionen des Latent-Spaces (\textit{Disentangled}-VAE). Im Allgemeinen sind die Rekonstruktionen des VAEs aber unscharf. Dies zeigt sich besonders bei harten Übergängen. 
Im Few-Shot Szenario zeigte sich auch hier eine konsistente Verbesserung des F-Scores. Außerdem konnte beobachtet werden, dass das selbst-überwachte Trainieren eines Single-VAEs bessere Ergebnisse erzielt, da wesentlich mehr Daten genutzt werden können.

\pagebreak

Zusammenfassend haben Variational Autoencoder ein großes Potenzial, reichen alleine jedoch nicht aus um herkömmliche Data Augmentation Techniken vollständig zu ersetzen. Das Verfahren ist im Allgemeinen äußerst Hyperparameter sensitiv. Die Encoder-Decoder Struktur bietet jedoch enorme Vorteile, da kontrollierbare Augmentationen im Latent-Space durchführbar sind. Mögliche Lösungsansätze für die Probleme und weitere Verbesserungen sind im folgenden Ausblick ausgeführt.
