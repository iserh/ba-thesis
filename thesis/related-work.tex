\chapter{Wissenschaftlicher Hintergrund}
In den letzten Jahren wurden bereits einige Arbeiten zu der Augmentation von Daten über Generative Modelle veröffentlicht. Einige dieser Arbeiten stellen Konzepte vor, welche auch in dieser Arbeit verwendet werden. Diese werden im Folgenden erläutert.\\

\cite{Jorge2018} untersuchen empirisch, wie die Verwendung von Variational Autoencoder zur Data Augmentation, die Klassifikation auf dem MNIST und Omniglot Datensatz verbessert. Außerdem stellen sie einige Methoden des Latent-Space Samplings vor, welche auch in dieser Arbeit aufgegriffen werden. \cite{Garay-Maestre2019} behandeln zusätzlich die Verbesserungen, die sich auf reduzierten Partitionen von MNIST erzielen lassen. Zudem wird die Verwendung von einem seperaten Modell je Klasse motiviert. Dieser Ansatz wird in der vorliegenden Arbeit aufgegriffen und auf weitere, unter anderem numerische Daten angewendet. \\

\cite{Moreno-Barea2020} beschäftigen sich in ihrer Arbeit mit der Anwendung verschiedener Generativer Modelle auf kleinen Datensätzen. Im Gegensatz zu den anderen Arbeiten in diesem Bereich, werden hier keine Bilddaten untersucht. Stattdessen werden numerische Daten mit teils diskreten Attributen behandelt. Außerdem stellen sie einen Data Augmentation Prozess vor, welcher zusätzlich eine Filterung der generierten Daten nutzt, um unplausible Beispiele auszusortieren. Damit soll die Performanz weiter verbessert werden. Da hier allerdings mehrere Generative Ansätze untersucht werden, macht der Variational Autoencoder nur einen Teil dieser Arbeit aus. Dieser wird in der vorliegenden Arbeit genauer analysiert. \\

\cite{Higgins2017} schlagen eine $\beta$ Erweiterung zum Variational Autoencoder vor, welche die Korrelation erlernter Merkmale beeinflusst. In der Arbeit werden ausschließlich Bilddaten untersucht. Der Fokus der Autoren liegt auf der Qualität der generierten Bilder und der unabhängigen Modifikation von Merkmalen. In der vorliegenden Bachelorarbeit wird darüber hinaus die Auswirkungen auf die Klassifikationsaufgabe für sowohl Bild-, als auch numerische Daten, behandelt. \\

\pagebreak
\cite{goodfellow2014generative} stellen mit "Generative-Adversarial-Networks" eine weitere Architektur eines generativen Modells vor. Diese bietet einige Vorteile gegenüber Variational Autoencodern. Unter anderem werden sehr viel schärfere Bilder generiert, dafür erlauben sie jedoch keine Kontrolle über die generierten Beispiele. Ein Vergleich zu dieser Methode ist einer der zu untersuchenden Aspekte dieser Arbeit.
