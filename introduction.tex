\chapter{Einführung}
Neuronale Netzwerke erwiesen sich in den letzten Jahren als eine der mächtigsten Ansätze im Bereich des maschinellen Lernens. Durch immer tiefere Modellarchitekturen konnten auch komplexere Aufgaben gelöst werden (z.B. AlphaFold, \cite{Senior2020}) . Insbesondere für die Aufgabe der Klassifikation dominieren Neuronale Netze den \textit{State-of-the-Art}. Dazu werden üblicherweise große Datenmengen benötigt, welche in vielen Szenarien jedoch nicht zur Verfügung stehen. Im Extremfall sind nur wenige Beispiele pro Klasse vorhanden (\textit{Few-Shot Learning}). Die Gesichtserkennung, wie sie mittlerweile in den meisten modernen Smartphones zu finden ist, stellt ein Beispiel für einen solchen Fall dar. Ein Möglichkeit dieses Problem zu lösen bietet das künstliche Erweitern der gegebenen Daten. \\

Data Augmentation hat sich als eine der Standard Methoden im Umgang mit kleinen oder imbalancierten Datensätzen etabliert. Üblicherweise werden dazu Beispiele aus dem Datensatz über unterschiedliche Transformationen (z.B. Rotation, Translation, Skalierung) augmentiert. Mit Hilfe dieser Methode konnten bemerkenswerte Ergebnisse, insbesondere in der Bildverarbeitung, erzielt werden. Dieser Prozess baut jedoch häufig auf Expertenwissen darüber auf, welche Transformationen für das Erlernen der Aufgabe hilfreich sind. Ohne dieses Expertenwissen können unplausible Beispiele entstehen, welche die Performanz negativ beeinträchtigen. \\

Alternativ bieten Generative Modelle eine effiziente Methode für die automatisierte Erzeugung von Daten. Mittels tiefer konvolutionaler Netzwerk Architekturen wird eine immer bessere Qualität der generierten Daten erreicht. Somit stellen sie inzwischen eine effiziente Alternative zu den klassischen Data Augmentation Methoden dar. Zudem bieten sie den Vorteil, dass im Allgemeinen kein Expertenwissen benötigt wird. So zeigten sie in verschiedensten Bereichen beachtliche Ergebnisse (z.B. GPT-3, \cite{brown2020language}). \\

In der vorliegenden Arbeit wird der Variational Autoencoder (VAE) als generatives Modell betrachtet. Der übliche Nutzen von Autoencodern liegt in einer semantischen Repräsentation der Eingabe. Dies wird über eine Encoder-Decoder Architektur erreicht. Variational Autoencoder erweitern diese Struktur, um eine probabilistische Repräsentation der Eingabe in einem \textit{Latent-Space} zu erzeugen. Diese Arbeit teilt sich in einen theoretischen und einen praktischen Fokus auf. Der theoretische Fokus liegt in der Analyse der Latent-Space Struktur. Insbesondere wird sich mit der Fragestellung beschäftigt, wie diese für die Erzeugung neuer Daten verwendet werden kann. Der praktische Fokus liegt auf der Evaluation verschiedener VAE basierter Data Augmentation Ansätze auf unterschiedlichen Datensätzen. Diese beinhalten sowohl Bild-, als auch numerische Daten. Zudem wird der Einfluss der VAE basierten Ansätze in Few-Shot Szenarien untersucht. Abschließend werden diese Verfahren mit anderen generativen Methoden verglichen.
